\documentclass[12pt,t]{beamer}
\usepackage{graphicx}
\usepackage{tikz}
\setbeameroption{hide notes}
\setbeamertemplate{note page}[plain]
\usepackage{listings}

% header.tex: boring LaTeX/Beamer details + macros

% get rid of junk
\usetheme{default}
\beamertemplatenavigationsymbolsempty
\hypersetup{pdfpagemode=UseNone} % don't show bookmarks on initial view


% font
\usepackage{fontspec}
\setsansfont
  [ ExternalLocation = fonts/ ,
    UprightFont = *-regular ,
    BoldFont = *-bold ,
    ItalicFont = *-italic ,
    BoldItalicFont = *-bolditalic ]{texgyreheros}
\setbeamerfont{note page}{family*=pplx,size=\footnotesize} % Palatino for notes
% "TeX Gyre Heros can be used as a replacement for Helvetica"
% I've placed them in fonts/; alternatively you can install them
% permanently on your system as follows:
%     Download http://www.gust.org.pl/projects/e-foundry/tex-gyre/heros/qhv2.004otf.zip
%     In Unix, unzip it into ~/.fonts
%     In Mac, unzip it, double-click the .otf files, and install using "FontBook"

% named colors
\definecolor{offwhite}{RGB}{255,250,240}
\definecolor{gray}{RGB}{155,155,155}

\ifx\notescolors\undefined % slides
  \definecolor{foreground}{RGB}{255,255,255}
  \definecolor{background}{RGB}{24,24,24}
  \definecolor{title}{RGB}{107,174,214}
  \definecolor{subtitle}{RGB}{102,255,204}
  \definecolor{hilit}{RGB}{102,255,204}
  \definecolor{vhilit}{RGB}{255,111,207}
  \definecolor{codehilit}{RGB}{255,111,207}
  \definecolor{lolit}{RGB}{155,155,155}
\else % notes
  \definecolor{background}{RGB}{255,255,255}
  \definecolor{foreground}{RGB}{24,24,24}
  \definecolor{title}{RGB}{27,94,134}
  \definecolor{subtitle}{RGB}{22,175,124}
  \definecolor{hilit}{RGB}{122,0,128}
  \definecolor{vhilit}{RGB}{255,0,128}
  \definecolor{codehilit}{RGB}{24,24,24}
  \definecolor{lolit}{RGB}{95,95,95}
\fi
\definecolor{nhilit}{RGB}{128,0,128}  % hilit color in notes
\definecolor{nvhilit}{RGB}{255,0,128} % vhilit for notes

\newcommand{\hilit}{\color{hilit}}
\newcommand{\vhilit}{\color{vhilit}}
\newcommand{\nhilit}{\color{nhilit}}
\newcommand{\nvhilit}{\color{nvhilit}}
\newcommand{\lolit}{\color{lolit}}

% use those colors
\setbeamercolor{titlelike}{fg=title}
\setbeamercolor{subtitle}{fg=subtitle}
\setbeamercolor{institute}{fg=lolit}
\setbeamercolor{normal text}{fg=foreground,bg=background}
\setbeamercolor{item}{fg=foreground} % color of bullets
\setbeamercolor{subitem}{fg=lolit}
\setbeamercolor{itemize/enumerate subbody}{fg=lolit}
\setbeamertemplate{itemize subitem}{{\textendash}}
\setbeamerfont{itemize/enumerate subbody}{size=\footnotesize}
\setbeamerfont{itemize/enumerate subitem}{size=\footnotesize}

% page number
\setbeamertemplate{footline}{%
    \raisebox{5pt}{\makebox[\paperwidth]{\hfill\makebox[20pt]{\lolit
          \scriptsize\insertframenumber}}}\hspace*{5pt}}

% add a bit of space at the top of the notes page
\addtobeamertemplate{note page}{\setlength{\parskip}{12pt}}

% default link color
\hypersetup{colorlinks, urlcolor={hilit}}

\ifx\notescolors\undefined % slides
  % set up listing environment
  \lstset{language=bash,
          basicstyle=\ttfamily\scriptsize,
          frame=single,
          commentstyle=,
          backgroundcolor=\color{darkgray},
          showspaces=false,
          showstringspaces=false
          }
\else % notes
  \lstset{language=bash,
          basicstyle=\ttfamily\scriptsize,
          frame=single,
          commentstyle=,
          backgroundcolor=\color{offwhite},
          showspaces=false,
          showstringspaces=false
          }
\fi

% a few macros
\newcommand{\bi}{\begin{itemize}}
\newcommand{\bbi}{\vspace{24pt} \begin{itemize} \itemsep8pt}
\newcommand{\ei}{\end{itemize}}
\newcommand{\ig}{\includegraphics}
\newcommand{\subt}[1]{{\footnotesize \color{subtitle} {#1}}}
\newcommand{\ttsm}{\tt \small}
\newcommand{\ttfn}{\tt \footnotesize}
\newcommand{\figh}[2]{\centerline{\includegraphics[height=#2\textheight]{#1}}}
\newcommand{\figw}[2]{\centerline{\includegraphics[width=#2\textwidth]{#1}}}

%%%%%%%%%%%%%%%%%%%%%%%%%%%%%%%%%%%%%%%%%%%%%%%%%%%%%%%%%%%%%%%%%%%%%%
% end of header
%%%%%%%%%%%%%%%%%%%%%%%%%%%%%%%%%%%%%%%%%%%%%%%%%%%%%%%%%%%%%%%%%%%%%%

% title info

\title{Forest of Eden}
\author{\href{https://github.com/aaronpeikert/}{Aaron Peikert}}
\institute{Humboldt{\textendash}Universität zu Berlin}
\date{
\scriptsize {\lolit Slides:} \href{https://github.com/aaronpeikert/ForestEden}{\tt \scriptsize
  \color{foreground} https://github.com/aaronpeikert/ForestEden}
}


\begin{document}

% title slide
{
{
\setbeamertemplate{footline}{} % no page number here
\frame{
  \titlepage

  \vfill \hfill \includegraphics[height=6mm]{Figs/cc-zero.png} \vspace*{-1cm}
  
  \note{These Slides are for a short Talk (10min) intended for a beginner audience where no statistical knowledge can be expected. 

    Source: {\tt https://github.com/aaronpeikert/ForestEden}
}}
}

\begin{frame}[c]{The Great Divide}
  \begin{center}
  \large
  ``There are two kinds of people:\\
  those who divide everything in into two kinds\\
  and those who don’t."
  \end{center}
  \hfill {\textendash} (paraphrasing) \lolit \href{http://hdl.handle.net/2027/mdp.39015032024203?urlappend=\%3Bseq=203}{Robert Benchley}
  \note{
    This epigram captures the unreasonable assumption behind trees that everyhing is a binary class. At the same time a lot of things can be captured in two classes. This fundamental assumption is in all, but the most trivial cases, false. However it proves to be most versatile and is what centrally drives decision trees.
  }
\end{frame}

\begin{frame}[c]{Goal}
  \begin{center}
  \large
  Let's stick to this binary world\\
  \note{
    Imagine our goal is to predict to which kind a person belongs. For that purpose we get information about that person.\\
    Translated in ML slang this is a classification task, where the class of an instance is to be predicted by certain features. 
  }
  \end{center}
\end{frame}

\begin{frame}[c]{The Great Divide}
  \large
  \onslide<1-|handout:1>\textcolor<5-|handout:0>{lolit}{If \textcolor<2,3,4|handout:1->{lolit}{condition}, then \textcolor<3,4|handout:1->{nvhilit}{class}, else \textcolor<4|handout:1->{vhilit}{other class}.} \\
  \onslide<5-|handout:1> \textcolor<handout:1->{nvhilit}{Class} and \textcolor<handout:1->{vhilit}{other class} can be again another condition.\\
  \onslide<6-|handout:1>Resulting in \textcolor<handout:1->{hilit}{yet another class}.\\
  \onslide<7|handout:0>\Huge Confused?
  \note{
    These simple if-condition-then-statements are natural to use. At the same time by combining many of such statements, complicated relations can be captured.
  }
\end{frame}

\begin{frame}[fragile, c]{Metaphor}
  \tikzset{
    treenode/.style = {shape=rectangle, rounded corners,
                       draw, align=center,
                       text=background,
                       color=lolit},
    root/.style     = {treenode},
    env/.style      = {treenode},
    level 1/.style={sibling distance=7em},
    level 2/.style={sibling distance=5em},
    level 3/.style={sibling distance=3em},
    level 4/.style={sibling distance=7em},
    class1/.style      = {treenode, color=hilit},
    class2/.style      = {treenode, color=nvhilit},
    dummy/.style    = {circle,draw}
  }
  \begin{tikzpicture}
    [
      grow                    = right,
      sibling distance        = 3em,
      level distance          = 6em,
      edge from parent/.style = {draw, -latex},
      every node/.style       = {font=\footnotesize},
      sloped
    ]
    \node [root] {initial condition}
      child { node [env] {condition 1} 
        child{ node [env]{condition 1.2}
          child{ node [class1]{class 1}}
          child{ node [class2]{class 2}}}
        child{ node [env] {condition 1.1}
          child{ node [class1]{class 1}}
          child{ node [class2]{class 2}}}}
      child { node [class1] {class 1}};
  \end{tikzpicture}
    \note{
    In the form of trees, really complicated statements can be communicated effectively. By traversing down the tree, taking turns at the nodes, the class at the end note is assigned. 
  }
\end{frame}


\begin{frame}[c]{Garden Eden of Algorythms}
  \large
  \begin{center}
  \only<1|handout:0>{Imagine growing these trees automaticly\\}
  \only<2|handout:1>{``With Decision Trees you can have your cake and eat it too."\\}
  \Large
  \only<3|handout:0>{Transparent}
  \only<4|handout:0>{Easy to understand}
  \only<5|handout:0>{Natural formulation}
  \only<6|handout:0>{Dirt cheap}
  \only<7|handout:0>{High capacity}
  \only<8|handout:0>{High generalizability}
  \only<9|handout:0>{Can model (almost) everything}
  \end{center}
  \note{
  Decision Trees have the following very fortunate properties:
  \begin{itemize}
  \item Transparent, all ``parameters" have a direkt meaning
  \item Easy to understand, the tree can be translated in ``If \ldots then \ldots" \textendash Statements
  \item Natural formulation, these statements are in much use. They are often employed to convey expert knowledge to non-experts.
  \item Dirt cheap in terms of computational costs
  \item High capacity or high generalizability, whatever is more appropriate
  \item Many data generating processes can be fitted, depending on the choosen capacity even all
  \end{itemize}
  }
\end{frame}

\begin{frame}[c]{Path to Garden Eden}
  \onslide<1-2|handout: 1>1. How can these \textcolor{hilit}{conditions} be constructed?\\
  \onslide<2|handout: 1>2. In which  \textcolor{vhilit}{order} should they apply?\\
  \onslide<3|handout: 0>The path out of \textcolor{lolit}{Garden Eden}\\is the path to \textcolor{lolit}{Forest Eden}.\\ 
  \note{
    In order to construct a tree automatically one needs to find an algorythm to determine which conditions are to be used and in which order.
  }
\end{frame}

\begin{frame}[c]{Rolemodel Devil}
  \Large
  \begin{center}
    \only<1|handout: 0>{Do it like the \textcolor{lolit}{Devil}\\}
    \only<2|handout: 0>{1. Seek \textcolor{hilit}{differences} and \textcolor{hilit}{inequality}\\}
    \only<3>{2. Be \textcolor{vhilit}{greedy}\\}
    \only<4>{1. differences = \textcolor{hilit}{conditions}\\
             2. greedy = \textcolor{vhilit}{order}}
  \end{center}
  \note{
    Trees often change fundamentally even when the input data is only slightly disturbed. Even though this is often not desirable, this characteristic can be used to our advantage.
  }
\end{frame}

\begin{frame}[c]{Moody Devil}
  \begin{center}
    \onslide<1-2>Devil = moody
    \onslide<2> = 
    \onslide<2-3>high entropy 
    \onslide<3> = 
    \onslide<3>sometimes useful?\\
  \end{center}
  \note{
    Trees often change fundamentally even when the input data is only slightly disturbed. Even though this is often not desirable, this characteristic can be used to grow many different trees. These ``forests" are created on purpose by slightly disturbing the training data. They can better differentiate between noise and systematics.
  }
\end{frame}

\begin{frame}[fragile]{The coin's other side}
  \begin{center}
    \onslide<1-2>{imagine there are magical coins \\}
    \onslide<2>{you are allowed to toss them 10 times \\}
    \only<3->{
      \onslide<3->Which estimate can you trust the most?\\
      \onslide<4->Coin: $\langle 0.5\rangle$\\
      \onslide<5->Pair: $\langle 0.3, 0.7 \rangle$\\
      \hspace{10mm}
    }
  \end{center}
\end{frame}

\begin{frame}[fragile]
  \begin{center}
  \begin{minipage}{\textwidth}
    \begin{lstlisting}
      k <- 10
      n <- 10000
      coin <- rbinom(n, k, .5)/k
      pair <- cbind(rbinom(n, k, .3), rbinom(n, k, .7))/k
      pair <- rowMeans(pair)
      experiment <- data.frame(coin, pair)
     \end{lstlisting}
   \end{minipage}
   \includegraphics{./Figs/coin}
   \end{center}
\end{frame}

\begin{frame}
  \includegraphics{./Figs/coin_cor}
\end{frame}

\begin{frame}[c]{Key Points}
  \onslide<1-2> 1. Variance can be reduced by sampling more...\\
  \onslide<2> 2. ...when samples are fairly \textcolor{hilit}{uncorrelated}.\\
  \centering
  \onslide<3> \Large How to build \textcolor{hilit}{uncorrelated} trees?
\end{frame}

\begin{frame}[c]{Pertubating Data}
    \onslide<1-2>Trees are volatile due to \textcolor{hilit}{greediness} \& \textcolor{vhilit}{maximizing}\\
    \onslide<2>slightly different data → drastically different trees\\
    \onslide<3->\textcolor{lolit}{(Re-)}sampling \textcolor{hilit}{cases}\\
    \onslide<4->sampling \textcolor{lolit}{(transformed)} \textcolor{hilit}{variables}\\
\end{frame}

\begin{frame}[c]{Pertubating Trees}
    \onslide<1->pertubate data at each split\\
    \onslide<2->randomize subtree order\\
    \onslide<3->randomize splitpoints\\
\end{frame}

\begin{frame}[c]{The classic}
    Leo Breiman (2001) proposed:
    resampling cases for trees \onslide<2->{→ \textcolor<3>{hilit}{b}\textcolor<3>{lolit}{ootstrap} \textcolor<3>{hilit}{agg}\textcolor<3>{lolit}{egrat}\textcolor<3>{hilit}{ing }}\newline
    sample variables for splits
\end{frame}

\begin{frame}

\Large

\vspace{10mm}

\scriptsize {\lolit Source:} \href{https://github.com/aaronpeikert/ForestEden}{\tt \scriptsize
  \color{foreground} https://github.com/aaronpeikert/ForestEden} \quad
\includegraphics[height=5mm]{Figs/cc-zero.png}

\vspace{10mm}

\scriptsize {\lolit Slidedesign:} \href{https://github.com/kbroman/Talk_ReproRes}{\tt \scriptsize
  \color{foreground} shamelessly copied from \emph{Karl Broman} \quad}

\vspace{10mm}

\scriptsize {\lolit Github:} \href{https://github.com/aaronpeikert/}{\tt \color{foreground} github.com/aaronpeikert/}

\vspace{10mm}

\scriptsize {\lolit Mail:} \tt aaron.peikert@hu-berlin.de

\note{
  Here's where you can find me, as well as the slides for this talk.
}
\end{frame}

\end{document}
